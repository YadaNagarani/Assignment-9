\documentclass[journal,12pt,twocolumn]{IEEEtran}

\usepackage{setspace}
\usepackage{gensymb}

\singlespacing


\usepackage[cmex10]{amsmath}

\usepackage{amsthm}

\usepackage{mathrsfs}
\usepackage{txfonts}
\usepackage{stfloats}
\usepackage{bm}
\usepackage{cite}
\usepackage{cases}
\usepackage{subfig}

\usepackage{longtable}
\usepackage{multirow}

\usepackage{enumitem}
\usepackage{mathtools}
\usepackage{steinmetz}
\usepackage{tikz}
\usepackage{circuitikz}
\usepackage{verbatim}
\usepackage{tfrupee}
\usepackage[breaklinks=true]{hyperref}
\usepackage{graphicx}
\usepackage{tkz-euclide}
\usepackage{float}

\usetikzlibrary{calc,math}
\usepackage{listings}
    \usepackage{color}                                            %%
    \usepackage{array}                                            %%
    \usepackage{longtable}                                        %%
    \usepackage{calc}                                             %%
    \usepackage{multirow}                                         %%
    \usepackage{hhline}                                           %%
    \usepackage{ifthen}                                           %%
    \usepackage{lscape}     
\usepackage{multicol}
\usepackage{chngcntr}

\DeclareMathOperator*{\Res}{Res}

\renewcommand\thesection{\arabic{section}}
\renewcommand\thesubsection{\thesection.\arabic{subsection}}
\renewcommand\thesubsubsection{\thesubsection.\arabic{subsubsection}}

\renewcommand\thesectiondis{\arabic{section}}
\renewcommand\thesubsectiondis{\thesectiondis.\arabic{subsection}}
\renewcommand\thesubsubsectiondis{\thesubsectiondis.\arabic{subsubsection}}


\hyphenation{op-tical net-works semi-conduc-tor}
\def\inputGnumericTable{}                                 %%

\lstset{
%language=C,
frame=single, 
breaklines=true,
columns=fullflexible
}
\begin{document}
\newtheorem{theorem}{Theorem}[section]
\newtheorem{problem}{Problem}
\newtheorem{proposition}{Proposition}[section]
\newtheorem{lemma}{Lemma}[section]
\newtheorem{corollary}[theorem]{Corollary}
\newtheorem{example}{Example}[section]
\newtheorem{definition}[problem]{Definition}

\newcommand{\BEQA}{\begin{eqnarray}}
\newcommand{\EEQA}{\end{eqnarray}}
\newcommand{\define}{\stackrel{\triangle}{=}}
\bibliographystyle{IEEEtran}
\providecommand{\mbf}{\mathbf}
\providecommand{\pr}[1]{\ensuremath{\Pr\left(#1\right)}}
\providecommand{\qfunc}[1]{\ensuremath{Q\left(#1\right)}}
\providecommand{\sbrak}[1]{\ensuremath{{}\left[#1\right]}}
\providecommand{\lsbrak}[1]{\ensuremath{{}\left[#1\right.}}
\providecommand{\rsbrak}[1]{\ensuremath{{}\left.#1\right]}}
\providecommand{\brak}[1]{\ensuremath{\left(#1\right)}}
\providecommand{\lbrak}[1]{\ensuremath{\left(#1\right.}}
\providecommand{\rbrak}[1]{\ensuremath{\left.#1\right)}}
\providecommand{\cbrak}[1]{\ensuremath{\left\{#1\right\}}}
\providecommand{\lcbrak}[1]{\ensuremath{\left\{#1\right.}}
\providecommand{\rcbrak}[1]{\ensuremath{\left.#1\right\}}}
\theoremstyle{remark}
\newtheorem{rem}{Remark}
\newcommand{\sgn}{\mathop{\mathrm{sgn}}}
\providecommand{\abs}[1]{\left\vert#1\right\vert}
\providecommand{\res}[1]{\Res\displaylimits_{#1}} 
\providecommand{\norm}[1]{\left\lVert#1\right\rVert}
%\providecommand{\norm}[1]{\lVert#1\rVert}
\providecommand{\mtx}[1]{\mathbf{#1}}
\providecommand{\mean}[1]{E\left[ #1 \right]}
\providecommand{\fourier}{\overset{\mathcal{F}}{ \rightleftharpoons}}
%\providecommand{\hilbert}{\overset{\mathcal{H}}{ \rightleftharpoons}}
\providecommand{\system}{\overset{\mathcal{H}}{ \longleftrightarrow}}
	%\newcommand{\solution}[2]{\textbf{Solution:}{#1}}
\newcommand{\solution}{\noindent \textbf{Solution: }}
\newcommand{\cosec}{\,\text{cosec}\,}
\providecommand{\dec}[2]{\ensuremath{\overset{#1}{\underset{#2}{\gtrless}}}}
\newcommand{\myvec}[1]{\ensuremath{\begin{pmatrix}#1\end{pmatrix}}}
\newcommand{\mydet}[1]{\ensuremath{\begin{vmatrix}#1\end{vmatrix}}}
\numberwithin{equation}{subsection}
\makeatletter
\@addtoreset{figure}{problem}
\makeatother
\let\StandardTheFigure\thefigure
\let\vec\mathbf
\renewcommand{\thefigure}{\theproblem}
\def\putbox#1#2#3{\makebox[0in][l]{\makebox[#1][l]{}\raisebox{\baselineskip}[0in][0in]{\raisebox{#2}[0in][0in]{#3}}}}
     \def\rightbox#1{\makebox[0in][r]{#1}}
     \def\centbox#1{\makebox[0in]{#1}}
     \def\topbox#1{\raisebox{-\baselineskip}[0in][0in]{#1}}
     \def\midbox#1{\raisebox{-0.5\baselineskip}[0in][0in]{#1}}
\vspace{3cm}
\title{ASSIGNMENT 9}
\author{Y.Nagarani}
\maketitle
\newpage
\bigskip
\renewcommand{\thefigure}{\theenumi}
\renewcommand{\thetable}{\theenumi}
Download all python codes from 
\begin{lstlisting}
https://github.com/Y.Nagarani/ASSIGNMENT9/tree/main/CODES
\end{lstlisting}
%
and latex-tikz codes from 
%
\begin{lstlisting}
https://github.com/Y.Nagarani/ASSIGNMENT9/tree/main
\end{lstlisting}
%
\section{Question No 2.15}
A manufacture produce nuts and bolts.It takes $1$ hour of work on a machine A and $3$ hours on machine B to produce a package of nuts.It takes $3$ hours on machine A and $1$ hour on machine B to produce a package of bolts. He earns a profit of Rs$17.50$ per package on nuts and Rs$7.00$ per package on bolts.How many packages of each should be produced each day so as to maximize his profit, if he operates his machines for at the most $12$ hours a day
%
\section{SOLUTION} 
\numberwithin{table}{section}
\begin{table}[!ht]
\centering
\resizebox{\columnwidth}{!}{\begin{tabular}{|c|c|c|c|} 
\hline
item & Machine A & Machine N & Earn profit \\
\hline
Nuts & 1 & 3  &  17.5  \\ 
\hline
Bolts & 3 & 1 &  7  \\ 
\hline
Hour's a day & 12 & 12 & \\ 
\hline
\end{tabular}}
\caption{Manufacturer produces nuts and bolts}
\label{tab:table1}
\end{table}
Let the number of nuts  be $x$ and the number of bolts be $y$  such that 
\begin{align}
x \geq 0 \\
y \geq 0 
\end{align}
According to the question,
\begin{align}
{x} + 3{y} \leq 12
\\
3{x} + {y} \leq 12
\end{align}
$\therefore$ Our problem is
\begin{align}
\max_{\vec{x}} Z &= \myvec{17.5 & 7}\vec{x}\\
s.t. \quad \myvec{1 & 3 \\ 3 & 1}\vec{x} &\preceq \myvec{12\\12} 
\end{align}
Lagrangian function is given by
\begin{equation}
\begin{aligned}
&L(\vec{x},\boldsymbol{\lambda}) \\ &= \myvec{17.5 & 7}\vec{x}+\lcbrak{\sbrak{\myvec{1 & 3}\vec{x}-12}} \\ &+ \sbrak{\myvec{3 & 1}\vec{x}-12}\\ &+ \sbrak{\myvec{-1 & 0}\vec{x}} +\rcbrak{\sbrak{\myvec{0 & -1}\vec{x}}}\boldsymbol{\lambda}
\end{aligned}
\end{equation}
where,
\begin{align}
\boldsymbol{\lambda} &= \myvec{\lambda_1 \\ \lambda_2 \\ \lambda_3 \\ \lambda_4 \\ \lambda_5 \\ \lambda_6}
\end{align}
Now,
\begin{align}
\nabla L(\vec{x},\boldsymbol{\lambda}) &= \myvec{17.5+ \myvec{1 & 3  & -1 & 0 }\boldsymbol{\lambda}\\ 7+\myvec{3 & 1 & 0 & -1}\boldsymbol{\lambda} \\ \myvec{1 & 3}\vec{x}-12 \\ \myvec{3 & 1}\vec{x}-12 \\  \myvec{-1 & 0}\vec{x} \\ \myvec{0 & -1}\vec{x}}
\end{align}
$\therefore$ Lagrangian matrix is given by
\begin{align}
\myvec{0 & 0 & 1 & 3 & -1 & 0 \\ 0 & 0 & 3 & 1  & 0 & -1 \\ 1 & 3 & 0 & 0 & 0 & 0 \\ 3 & 1 & 0 & 0 & 0 & 0  \\ -1 & 0 & 0 & 0 & 0 & 0  \\ 0 & -1 & 0 & 0 & 0 & 0 }\myvec{\vec{x} \\ \boldsymbol{\lambda} } &= \myvec{-17.5 \\ -7 \\ 12 \\ 12 \\ 0 \\0 }
\end{align}
Considering $\lambda_1,\lambda_2$ as only active multiplier,
\begin{align}
\myvec{0 & 0 & 1 & 3 \\ 0 & 0 & 3 & 1 \\ 1 & 3 & 0 & 0 \\ 3 & 1 & 0 & 0}\myvec{\vec{x}\\ \boldsymbol{\lambda}} &= \myvec{-17.5 \\ -7 \\ 12 \\ 12}
\end{align}
resulting in,
\begin{align}
\myvec{\vec{x} \\ \boldsymbol{\lambda}} &= \myvec{0 & 0 & 1 & 3 \\ 0 & 0 & 3 & 1 \\ 1 & 3 & 0 & 0 \\ 3 & 1 & 0 & 0}^{-1}\myvec{-17.5 \\ -7 \\ 12 \\ 12}
\\
\implies   \myvec{\vec{x} \\ \boldsymbol{\lambda}} &= \myvec{0 & 0 & \frac{-1}{8} & \frac{3}{8} \\ 0 & 0 & \frac{3}{8} & \frac{-1}{8} \\ \frac{-1}{8} & \frac{3}{8} & 0 & 0 \\ \frac{3}{8} & \frac{-1}{8} & 0 & 0}\myvec{-17.5 \\ -7 \\ 12 \\ 12}
\\
\implies \myvec{\vec{x} \\ \boldsymbol{\lambda}} &= \myvec{3 \\ 3 \\ -0.5 \\ -5.7 }
\end{align}
$\because \boldsymbol{\lambda}=\myvec{-0.5 \\ -5.7} \succ \vec{0} $
\\
$\therefore$ Optimal solution is given by
\begin{align}
    \vec{x} &= \myvec{3\\3} \\
    Z &= \myvec{17.5 & 7}\vec{x} \\
    &= \myvec{17.5 & 7}\myvec{3 \\ 3} \\
    &= 73.5
\end{align}
By using cvxpy in python ,
\begin{align}
    \vec{x}=\myvec{3\\3}\\
    Z = 73.49999997
\end{align}
Hence ,\boxed{x=3} Nuts and \boxed{y=3} Bolts should be used to maximum time Available profit \boxed{Z=73.5}.\\
\numberwithin{figure}{section}
\begin{figure}[!ht]
\centering
\includegraphics[width=\columnwidth]{download (2).png}
\caption{Graphical Solution}
\label{fig: Graphical Solution}	
\end{figure}
\end{document}