\documentclass[journal,12pt,twocolumn]{IEEEtran}

\usepackage{setspace}
\usepackage{gensymb}

\singlespacing


\usepackage[cmex10]{amsmath}

\usepackage{amsthm}

\usepackage{mathrsfs}
\usepackage{txfonts}
\usepackage{stfloats}
\usepackage{bm}
\usepackage{cite}
\usepackage{cases}
\usepackage{subfig}

\usepackage{longtable}
\usepackage{multirow}

\usepackage{enumitem}
\usepackage{mathtools}
\usepackage{steinmetz}
\usepackage{tikz}
\usepackage{circuitikz}
\usepackage{verbatim}
\usepackage{tfrupee}
\usepackage[breaklinks=true]{hyperref}
\usepackage{graphicx}
\usepackage{tkz-euclide}
\usepackage{float}

\usetikzlibrary{calc,math}
\usepackage{listings}
    \usepackage{color}                                            %%
    \usepackage{array}                                            %%
    \usepackage{longtable}                                        %%
    \usepackage{calc}                                             %%
    \usepackage{multirow}                                         %%
    \usepackage{hhline}                                           %%
    \usepackage{ifthen}                                           %%
    \usepackage{lscape}     
\usepackage{multicol}
\usepackage{chngcntr}

\DeclareMathOperator*{\Res}{Res}

\renewcommand\thesection{\arabic{section}}
\renewcommand\thesubsection{\thesection.\arabic{subsection}}
\renewcommand\thesubsubsection{\thesubsection.\arabic{subsubsection}}

\renewcommand\thesectiondis{\arabic{section}}
\renewcommand\thesubsectiondis{\thesectiondis.\arabic{subsection}}
\renewcommand\thesubsubsectiondis{\thesubsectiondis.\arabic{subsubsection}}


\hyphenation{op-tical net-works semi-conduc-tor}
\def\inputGnumericTable{}                                 %%

\lstset{
%language=C,
frame=single, 
breaklines=true,
columns=fullflexible
}
\begin{document}
\newtheorem{theorem}{Theorem}[section]
\newtheorem{problem}{Problem}
\newtheorem{proposition}{Proposition}[section]
\newtheorem{lemma}{Lemma}[section]
\newtheorem{corollary}[theorem]{Corollary}
\newtheorem{example}{Example}[section]
\newtheorem{definition}[problem]{Definition}

\newcommand{\BEQA}{\begin{eqnarray}}
\newcommand{\EEQA}{\end{eqnarray}}
\newcommand{\define}{\stackrel{\triangle}{=}}
\bibliographystyle{IEEEtran}
\providecommand{\mbf}{\mathbf}
\providecommand{\pr}[1]{\ensuremath{\Pr\left(#1\right)}}
\providecommand{\qfunc}[1]{\ensuremath{Q\left(#1\right)}}
\providecommand{\sbrak}[1]{\ensuremath{{}\left[#1\right]}}
\providecommand{\lsbrak}[1]{\ensuremath{{}\left[#1\right.}}
\providecommand{\rsbrak}[1]{\ensuremath{{}\left.#1\right]}}
\providecommand{\brak}[1]{\ensuremath{\left(#1\right)}}
\providecommand{\lbrak}[1]{\ensuremath{\left(#1\right.}}
\providecommand{\rbrak}[1]{\ensuremath{\left.#1\right)}}
\providecommand{\cbrak}[1]{\ensuremath{\left\{#1\right\}}}
\providecommand{\lcbrak}[1]{\ensuremath{\left\{#1\right.}}
\providecommand{\rcbrak}[1]{\ensuremath{\left.#1\right\}}}
\theoremstyle{remark}
\newtheorem{rem}{Remark}
\newcommand{\sgn}{\mathop{\mathrm{sgn}}}
\providecommand{\abs}[1]{\left\vert#1\right\vert}
\providecommand{\res}[1]{\Res\displaylimits_{#1}} 
\providecommand{\norm}[1]{\left\lVert#1\right\rVert}
%\providecommand{\norm}[1]{\lVert#1\rVert}
\providecommand{\mtx}[1]{\mathbf{#1}}
\providecommand{\mean}[1]{E\left[ #1 \right]}
\providecommand{\fourier}{\overset{\mathcal{F}}{ \rightleftharpoons}}
%\providecommand{\hilbert}{\overset{\mathcal{H}}{ \rightleftharpoons}}
\providecommand{\system}{\overset{\mathcal{H}}{ \longleftrightarrow}}
	%\newcommand{\solution}[2]{\textbf{Solution:}{#1}}
\newcommand{\solution}{\noindent \textbf{Solution: }}
\newcommand{\cosec}{\,\text{cosec}\,}
\providecommand{\dec}[2]{\ensuremath{\overset{#1}{\underset{#2}{\gtrless}}}}
\newcommand{\myvec}[1]{\ensuremath{\begin{pmatrix}#1\end{pmatrix}}}
\newcommand{\mydet}[1]{\ensuremath{\begin{vmatrix}#1\end{vmatrix}}}
\numberwithin{equation}{subsection}
\makeatletter
\@addtoreset{figure}{problem}
\makeatother
\let\StandardTheFigure\thefigure
\let\vec\mathbf
\renewcommand{\thefigure}{\theproblem}
\def\putbox#1#2#3{\makebox[0in][l]{\makebox[#1][l]{}\raisebox{\baselineskip}[0in][0in]{\raisebox{#2}[0in][0in]{#3}}}}
     \def\rightbox#1{\makebox[0in][r]{#1}}
     \def\centbox#1{\makebox[0in]{#1}}
     \def\topbox#1{\raisebox{-\baselineskip}[0in][0in]{#1}}
     \def\midbox#1{\raisebox{-0.5\baselineskip}[0in][0in]{#1}}
\vspace{3cm}
\title{ASSIGNMENT 9}
\author{Y.Nagarani}
\maketitle
\newpage
\bigskip
\renewcommand{\thefigure}{\theenumi}
\renewcommand{\thetable}{\theenumi}
Download all python codes from 
\begin{lstlisting}
https://github.com/Y.Nagarani/ASSIGNMENT9/tree/main/CODES
\end{lstlisting}
%
and latex-tikz codes from 
%
\begin{lstlisting}
https://github.com/Y.Nagarani/ASSIGNMENT9/tree/main
\end{lstlisting}
%
\section{Question No 2.15}
A manufacture produce nuts and bolts.It takes $1$ hour of work on a machine A and $3$ hours on machine B to produce a package of nuts.It takes $3$ hours on machine A and $1$ hour on machine B to produce a package of bolts. He earns a profit of Rs$17.50$ per package on nuts and Rs$7.00$ per package on bolts.How many packages of each should be produced each day so as to maximize his profit, if he operates his machines for at the most $12$ hours a day
%
\section{SOLUTION} 
Let,
\ x = No.of packages of nuts.
\\
\ y = No.of packages of bolts.
\begin{align}
 x,y\geq 0
\end{align}
And that is
\begin{align}
\ x+3{y}\leq12
\\
\ 3{x}+y\leq12
\end{align}
Which can be written as ,
\begin{align}
\myvec{-1 & -3 \\ -3 & -1}\vec{x} = \myvec{-12 \\ 12}
\end{align}
Let,${u_1}\geq 0 , {u_2}\geq 0$. This may be expressed as ,
\begin{align}
\Vec{u} = \myvec{{u_1} \\ {u_2}}\geq \vec{0}
\end{align}
Now ,we have
\begin{align}
\myvec{-1 & -3 \\ -3 & -1}\vec{x} = \myvec{-12 \\ -12} + \vec{u}
\\
\vec{x} = \myvec{-1 & -3 \\ -3 & -1}^{-1}\myvec{-12 \\ -12} + \myvec{-1 & -3 \\ -3 & -1}^{-1}\vec{u}
\\
\vec{x} = \myvec{3 \\ 3} - \frac{1}{8}\myvec{-1 & 3 \\ 3 & -1}\vec{u}
\end{align}
Thus , the solution of the system of inequalities can be determined graphically and the desired region is the shaded  triangle $\label{ref: Graphical Solution}$ which is represented in below fig 2.1
\numberwithin{figure}{section}
\begin{figure}[!ht]
\centering
\includegraphics[width=\columnwidth]{download (2).png}
\caption{Graphical Solution}
\label{fig: Graphical Solution}	
\end{figure}
\\
Now, we have to find profit maximum from graphical solution for nuts and bolts
\\
Therefore , profit maximum
\begin{align}
\ z = 17.5{x}+7{y}
\end{align}
Corner points are $A = (0,4)$ , $O = (3,3)$ and $D = (4,0)$
\\
Intersecting of two lines at a point (3,3) and that is maximum point.
\begin{align}
\ z = 17.5{3} + 7{3} = 73.5
\end{align}
Therefore , maximum value of z is Rs$73.5$ if be produce and sell $3$ units of nuts and $3$ units of bolts.
\end{document}